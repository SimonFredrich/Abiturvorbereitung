\documentclass[12pt, a4paper]{report}
\usepackage[utf8]{inputenc}
\usepackage[german]{babel}
\usepackage[T1]{fontenc}
\usepackage{amsmath}
\usepackage{amsfonts}
\usepackage{amssymb}

\title{Abiturvorbereitung Mathematik}
\author{Simon Fredrich}
\date{2020-2021}

\begin{document}

\begin{titlepage}
\maketitle
\end{titlepage}

\begin{abstract}
% Der Abstract ist eine kurze Beschreibung der Arbeit.
In dieser Arbeit werde ich den Stoff der 12. und 13. Klasse des Bereiches Mathematik zusammenfassen und mit Beispielen ausführen. Es handelt sich um den Mathematik Leistungskurs.
\end{abstract}

\chapter{Integralrechnung}

\section{Hauptsatz der Integral- und Differentialrechnung}
Ist $f$ eine im $I[a;b]$ stetige Funktion und F eine zu $f$ gehörende Stammfunktion so gilt:\\
\begin{equation}
\int_{a}^{b} f(x) dx = F(b)-F(a)
\end{equation}

\subsection{Beispiel}

\begin{equation}
\int_{0}^{2} x^2 dx = \left[\frac{x^3}{3}\right]_0^2=\frac{2^3}{3}-\frac{0^3}{3}=\frac{8}{3}
\end{equation}
Mit $C=-2$:
\begin{equation}
\int_{0}^{2} x^2 dx = \left[\frac{x^3}{3}-2\right]_0^2=\left(\frac{2^3}{3}-2\right)-\left(\frac{0^3}{3}-2\right)=\frac{8}{3}
\end{equation}

\section{Regeln}
Das Ermitteln unbestimmter Integrale.
\subsection{Potenzregel}

\subsection{Summenregel}
\subsection{Faktorregel}
\subsection{lineare Kettenregel/Substitution}

\section{Die Integralfunktion}
Dem bestimmten Integral kann bei Veränderung der oberen Integrationsgrenze b genau eine Zahl zugeordnet werden.
\begin{equation}
\int_0^b x^2 dx=\frac{b^3}{3}
\end{equation}
Dies ist das bestimmte Integral zur oberen Grenze b.


\section{Stammfunktionen}
Eine differenzierbare Funktion F, für die gilt $F'(x)=f(x)$ heißt Stammfunktion von f. \\
$\implies$ Integralfunktionen sind Stammfunktionen\\
Die \emph{Menge} aller Stammfunktionen einer Funktion $f$ heißt \emph{unbestimmtes Integral} von $f$.\\

\begin{equation}
\begin{aligned}
\int f(x) dx=F(x)+C \quad \textrm{and} \quad C\in \mathbb{R}
\end{aligned}
\end{equation}

\subsection{Beispiele}
\begin{enumerate}
\item $f(x)=6x$\\
$F_1(x)=3x^2$\\
$F_2(x)=3x^2+4$\\
$F_3(x)=3x^2-5$
\item $f(x)=7$\\
$F_1(x)=7x$\\
$F_2(x)=7x+16$\\
$F_3(x)=7x-3$
\end{enumerate}

\subsection{Satz}
$F_1$ und $F_2$ sind Stammfunktionen von $f$, dann ist $F_1-F_2$ eine konstante Funktion. Das heißt $F_1$ und $F_2$ unterscheiden sich nur um eine additive Konstante.

\subsection{Beweis}
$F_1$ und $F_2$ sind Stammfunktionen von $f$.\\
$\implies F_1' = f$ und $F_2' = f$\\
$\implies F_1'-F_2' = 0$\\
$\implies (F_1-F_2)' = 0$\\
Eine Funktion, deren Ableitung null ist, ist eine Konstante.\\
$\implies F_1-F_2 = C$




\end{document}