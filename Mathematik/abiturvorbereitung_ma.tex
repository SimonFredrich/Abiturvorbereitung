\documentclass[12pt, a4paper]{report}
\usepackage[utf8]{inputenc}
\usepackage[german]{babel}
\usepackage[T1]{fontenc}
\usepackage{amsmath, nccmath}
\usepackage{geometry}
\usepackage{amsfonts}
\usepackage{amssymb}
\usepackage{xcolor}

\title{Abiturvorbereitung Mathematik}
\author{Simon Fredrich}
\date{2020-2021}

\begin{document}

\begin{titlepage}
\maketitle
\end{titlepage}

\begin{abstract}
% Der Abstract ist eine kurze Beschreibung der Arbeit.
In dieser Arbeit werde ich den Stoff der 12. und 13. Klasse des Bereiches Mathematik zusammenfassen und mit Beispielen ausführen. Es handelt sich um den Mathematik Leistungskurs.
\end{abstract}

\chapter{Gebrochen-rationale Funktionen}

\section{Polstellen und Asymptoten}
Eine Funktion $f$ mit $f(x)=\frac{Z(x)}{N(x)}$ bei der $Z$ und $N$ Polynome sind und $N(x)\neq 0$ ist, heißt \textit{rationale} Funktion. Die maximale Definitionsmenge ist $\mathbb{D}_f = \mathbb{R} \textbackslash \{x|N(x) = 0\}$.

$f$ ist \textit{ganzrational}, wenn $f$ auf $\mathbb{D}_f$ als Polynom darstellbar ist $\left(f(x)=\frac{g(x)}{1}\right)$. Anschaulich heißt das, dass der Graph von $f$ nur Löcher im Funktionsgraphen hat. $f$ ist jedoch \textit{gebrochen-rational}, wenn:
\begin{enumerate}
\item
$\text{Grad von } Z(x) < \text{Grad von } N(x) \implies \textit{echt }\text{gebrochen-rational}$
\item
$\text{Grad von } Z(x) > \text{Grad von } N(x) \implies \textit{unecht }\text{gebrochen-rational}$
\end{enumerate}

In beiden Fällen lässt sich mir der Polynomdivision der ganzrationale und echt gebrochen-rationale Anteil feststellen.

\section{Beispiele}

\begin{fleqn}[\parindent]
\begin{enumerate}
\item
\begin{align*}
f(x)=\frac{x^3-x^2-26x-19}{x-5}
\end{align*}
\begin{align*}
\mathbb{D}_f = \{x\in \mathbb{R}; x \neq 5\}
\end{align*}
\begin{align*}
Z(5)=-49 \neq 0
\end{align*}
\begin{align*}
N(5)=0
\end{align*}
$x=5$ ist keine Nullstelle des Zählerpolynoms, weshalb man den Linearfaktor $(x-5)$ im Zähler nicht abspalten kann. $f$ ist \textit{gebrochen-rational}. Zählergrad 3 > Nennergrad 1. Die Funktion ist also \textit{unecht} gebrochen.
\begin{align*}
(x^3-x^2-26x-19):(x-5)=\color{blue}x^2+4x-6\color{red}-\frac{49}{x-5}
\end{align*}
\textcolor{blue}{ganzrationaler Anteil}\\
\textcolor{red}{echt gebrochen-rationaler Anteil}

\item
\begin{align*}
f(x)=\frac{x^3+4x^2+x-6}{x+2}
\end{align*}
\begin{align*}
\mathbb{D}_f = \mathbb{R}\{-2\}
\end{align*}
\begin{align*}
Z(-2)=0
\end{align*}
\begin{align*}
N(-2)=0
\end{align*}
Im Zähler kann der Linearfaktor $(x+2)$ abgespalten werden. $f$ ist demnach \textit{ganzrational}.
\begin{align*}
(x^3+4x^2+x-6):(x+2)=\color{blue}x^2+2x-3
\end{align*}
\textcolor{blue}{ganzrationaler Anteil}

Das durch die Definitionslücke veruhrsachte Loch im Graphen von $f$ kann geschlossen werden.
\begin{align*}
f(x)=x^2+2x-3
\end{align*}
\end{enumerate}
\end{fleqn}




\chapter{Integralrechnung}

\section{Hauptsatz der Integral- und Differentialrechnung}
Ist $f$ eine im $I[a;b]$ stetige Funktion und F eine zu $f$ gehörende Stammfunktion so gilt:\\
\begin{equation}
\int_{a}^{b} f(x) dx = F(b)-F(a)
\end{equation}

\subsection{Beispiele}

\begin{equation}
\int_{0}^{2} x^2 dx = \left[\frac{x^3}{3}\right]_0^2=\frac{2^3}{3}-\frac{0^3}{3}=\frac{8}{3}
\end{equation}
Mit $C=-2$ kämen wir auf das gleiche Ergebnis:
\begin{equation}
\int_{0}^{2} x^2 dx = \left[\frac{x^3}{3}-2\right]_0^2=\left(\frac{2^3}{3}-2\right)-\left(\frac{0^3}{3}-2\right)=\frac{8}{3}
\end{equation}

\section{Integrationsregeln}
Das Ermitteln unbestimmter Integrale.
\subsection{Potenzregel}
\begin{equation}
\int x^n dx = \frac{x^{n+1}}{n+1}+C\quad \land \quad n\neq-1\quad \land \quad n\in \mathbb{Z}\quad \land \quad  C\in \mathbb{R}
\end{equation}
\subsubsection{Beispiele}

\begin{fleqn}[\parindent]
\begin{equation}
\begin{split}
\int x^3 dx = \frac{x^{3+1}}{3+1}+C = \frac{x^{4}}{4}+C
\end{split}
\end{equation}
\begin{equation}
\begin{split}
\int \sqrt{x} dx = \int x^{\frac{1}{2}} dx = \frac{x^{\frac{1}{2}+1}}{\frac{1}{2}+1}+C=\frac{2}{3}x^{\frac{3}{2}}+C=\frac{2}{3}\sqrt{x}^{\frac{3}{2}}+C
\end{split}
\end{equation}
\begin{equation}
\begin{split}
\int adx = ax+C
\end{split}
\end{equation}
\begin{equation}
\begin{split}
\int 0dx = C
\end{split}
\end{equation}
\end{fleqn}

\subsection{Summenregel}

Man kann eine Summe gliedweise integrieren.
\begin{equation}
\int (f(x)+g(x)) dx = \int f(x)dx+\int g(x)dx=F(x)+G(x)
\end{equation}

\subsubsection{Beispiele}

\begin{fleqn}[\parindent]
\begin{multline}
\int (x^3 + 2x^{10}-80x^2)dx\\=
\int x^3dx + \int 2x^{10}dx - \int 80x^2dx = \frac{1}{4}x^4 + \frac{2}{11}x^{11} - \frac{80}{3}x^3 + C
\end{multline}
\begin{multline}
\int (\sqrt{x} + \sqrt[3]{x}) dx = \int (x^{\frac{1}{2}}+x^{\frac{1}{3}}) dx =\int x^{\frac{1}{2}}+\int x^{\frac{1}{3}} dx\\ =
\frac{2}{3}x^{\frac{3}{2}} + \frac{3}{4}x^{\frac{4}{3}} + C = \frac{2}{3}\sqrt{x}^3 + \frac{3}{4}\sqrt{x}^4 +C
\end{multline}
\end{fleqn}

\subsection{Faktorregel}
Ein konstanter Faktor bleibt beim Integrieren erhalten.

\begin{equation}
\int a\cdot f(x)dx = a\cdot \int f(x)dx \quad \land \quad (a \in \mathbb{R})
\end{equation}

\subsubsection{Beispiele}
\begin{fleqn}[\parindent]
\begin{enumerate}
\item
\begin{align*}
\int 14x^8dx = 14\int x^8dx = 14\cdot \frac{1}{9}x^9 + C
\end{align*}

\item
\begin{align*}
\int (14x^8 + 5x^9)dx = 14\cdot \int x^8dx + 5\cdot \int x^9 = 14\cdot \frac{1}{9}x^9 + 5\cdot \frac{1}{10}x^{10} + C\\
= \frac{14}{9}x^9 + \frac{1}{2}x^{10} + C
\end{align*}
\end{enumerate}
\end{fleqn}

\subsection{lineare Kettenregel/Substitution}
Für $a, b \in \mathbb{R}, a \neq 0$ gilt:
\begin{equation}
\int f(ax+b)dx=\frac{1}{a}F(ax+b)
\end{equation}

\subsubsection{Beispiele}

\begin{fleqn}[\parindent]
\begin{enumerate}
\item
\begin{align*}
\int (5x+1)^2dx = \frac{1}{5}\cdot \frac{1}{3} (5x+1)^3 = \frac{(5x+1)^3}{15} + C
\end{align*}
\item
\begin{align*}
\int (18x+144)^6dx = \frac{1}{18}\cdot \frac{1}{7} (5x+1)^7 = \frac{(18x+144)^7}{126} + C
\end{align*}
\end{enumerate}
\end{fleqn}

\subsection{Exponentielle Integration}
Folgende Ableitungsregeln sind in diesem Zusammenhang wichtig:
\begin{equation}
({\rm e}^x)' = {\rm e}^x, \quad (a^x)' = \ln a \cdot a^x
\end{equation}

Es ergeben sich im Umkehrschluss folgende Regeln:

\begin{equation}
\int {\rm e}^xdx = {\rm e}^x + C
\end{equation}
\begin{equation}
\int a^xdx = \frac{1}{\ln a} \cdot a^x + C
\end{equation}

\subsubsection{Beispiele}

\begin{fleqn}[\parindent]
\begin{enumerate}
\item
\begin{align*}
\int {\rm e}^{-2x}dx = \frac{1}{-2}{\rm e}^{-2x} + C
\end{align*}
\item
\begin{align*}
\int \left(\frac{5}{2}{\rm e}^{-4x}\right)dx = \frac{5}{2}\cdot \frac{1}{-4} \cdot {\rm e}^{-4x} + C = -\frac{5}{8}{\rm e}^{-4x} + C
\end{align*}
\end{enumerate}
\end{fleqn}

\subsection{Trigonometrische Integration}
Folgende Ableitungsregeln sind in diesem Zusammenhang wichtig:
\begin{equation}
(\sin x)' = \cos x, \quad (\cos x)' = -\sin x
\end{equation}

Daraus ergeben sich im Umkehrschluss folgende Regeln:
\begin{equation}
\int (\sin x)dx = -\cos x + C
\end{equation}
\begin{equation}
\int (\cos x) dx = \sin x +C
\end{equation}

\subsubsection{Beispiele}
\begin{fleqn}[\parindent]
\begin{enumerate}
\item
\begin{align*}
\int 4\sin(2x)dx = -4\cdot \frac{1}{2} \cdot \cos(2x) + C =-2 \cdot \cos(2x) + C
\end{align*}
\item
\begin{align*}
\int 4\cos(2x)dx = 4\cdot \frac{1}{2} \cdot \sin(2x) + C =2 \cdot \sin(2x) + C
\end{align*}
\end{enumerate}
\end{fleqn}

\section{Die Integralfunktion}
Dem bestimmten Integral kann bei Veränderung der oberen Integrationsgrenze b genau eine Zahl zugeordnet werden.
\begin{equation}
\int_0^b x^2 dx=\frac{b^3}{3}
\end{equation}
Dies ist das bestimmte Integral zur oberen Grenze b.


\section{Stammfunktionen}
Eine differenzierbare Funktion F, für die gilt $F'(x)=f(x)$ heißt Stammfunktion von f. \\
$\implies$ Integralfunktionen sind Stammfunktionen\\
Die \emph{Menge} aller Stammfunktionen einer Funktion $f$ heißt \emph{unbestimmtes Integral} von $f$.\\

\begin{equation}
\begin{aligned}
\int f(x) dx=F(x)+C \quad \land \quad C\in \mathbb{R}
\end{aligned}
\end{equation}

\subsection{Beispiele}
\begin{enumerate}
\item $f(x)=6x$\\
$F_1(x)=3x^2$\\
$F_2(x)=3x^2+4$\\
$F_3(x)=3x^2-5$
\item $f(x)=7$\\
$F_1(x)=7x$\\
$F_2(x)=7x+16$\\
$F_3(x)=7x-3$
\end{enumerate}

\subsection{Satz}
$F_1$ und $F_2$ sind Stammfunktionen von $f$, dann ist $F_1-F_2$ eine konstante Funktion. Das heißt $F_1$ und $F_2$ unterscheiden sich nur um eine additive Konstante.

\subsection{Beweis}
$F_1$ und $F_2$ sind Stammfunktionen von $f$.\\
$\implies F_1' = f$ und $F_2' = f$\\
$\implies F_1'-F_2' = 0$\\
$\implies (F_1-F_2)' = 0$\\
Eine Funktion, deren Ableitung null ist, ist eine Konstante.\\
$\implies F_1-F_2 = C$




\end{document}